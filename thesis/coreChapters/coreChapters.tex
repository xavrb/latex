\chapter{Core Chapters}

The structure of your thesis is up to you and your supervisor. Whatever you
do, do not consider the guidelines below as dogmas.

\section{Classic structure}

\begin{description}
  \item[Problem statement and motivation.]
  You describe in detail what problem the research is addressing, and what is
the motivation to address this problem. There is a concise and objective
statement of the research questions, hypotheses and goals. It is made clear
why these questions and goals are important and relevant to the world outside
the university (assuming it exists). You can already split the main research
question into subquestions in this chapter. This section also describes an
analysis of the problem: where does it occur and how, how often, and what are
the consequences? An important part is also to scope the research: what
aspects are included and what aspects are deliberately left out, and why?
  \item[Research method.]
  Here you describe the methods used to answer the research questions. A good
structure of this section often follows the subquestions by providing a method
for each. The research method needs a thorough motivation grounded in theory
in order to be acceptable. As a part of the method, you can introduce a number
of hypotheses --- these will be tested by the research, using the methods
described here. An important part of this section is validation. How will you
evaluate and validate the outcomes of the research?
  \item[Background and context.]
  This chapter contains all the information needed to put the thesis into
context. It is common to use a revised version of your literature survey for
this purpose. It is important to explicitly refer from your text to sources
you have used, they will be listed in your bibliography. For example, you can
write ``A small number of programming languages account for most language
use~\cite{MeyerovichR2013}'', where the following entry would be included in
your bibliography:
\begin{quote}
\cite{MeyerovichR2013} Leo A. Meyerovich and Ariel S. Rabkin. Empirical Analysis of Programming Language Adoption. In \emph{Proceedings of the 2013 ACM SIGPLAN International Conference on Object Oriented Programming Systems Languages and Applications}, OOPSLA, pages 1--18. ACM, 2013. \doi{10.1145/2509136.2509515}.
\end{quote}
Have a look at \autoref{sec:biblio} to learn more about citation.
  \item[Research.]
  This chapter reports on the execution of the research method as described in
an earlier chapter. If the research has been divided into phases, they are
introduced, reported on and concluded individually. If needed, this chapter
could be split up to balance out the sizes of all chapters.
  \item[Results.]
  This chapter presents and clarifies the results obtained during the
  research. The focus should be on the factual results, not the interpretation
  or discussion. Tables and graphics should be used to increase the clarity of
  the results where applicable.
  \item[Analysis and conclusions.]
  This chapter contains the analysis and interpretation of the results. The
  research questions are answered as best as possible with the results that
  were obtained. The analysis also discussed parts of the questions that were
  left unanswered.

  An important topic is the validity of the results. What methods of
  validation were used? Could the results be generalised to other cases? What
  threats to validity can be identified? There is room here to discuss the
  results of related scientific literature here as well. How do the results
  obtained here relate to other work, and what consequences are there? Did
  your approach work better or worse? Did you learn anything new compared to
  the already existing body of knowledge? Finally, what could you say in
  hindsight on the research approach by followed? What could have done better?
  What lessons have been learned? What could other researchers use from your
  experience? A separate section should be devoted to ``future work'', i.e.,
  possible extension points of your work that you have identified. Even other
  researchers should be able to use those as a starting point.
\end{description}

\section{Reporting on replications}

Here are the guidelines to report on replicated studies~\cite{Carver10}:

\begin{description}
  \item[Information about the original study]~\\
    \begin{description}
    \item[Research question(s)] that were the basis for the design
    \item[Participants,] their number and any other relevant characteristics
    \item[Design] as a graphical or textual description of the experimental design
    \item[Artefacts,] the description of them and/or links to the artefacts used
    \item[Context variables] as any important details that affected the design of the study or interpretation of the 
results
    \item[Summary of the results] in a brief overview of the major findings
    \end{description}
  %
  \item[Information about the replication]~\\
    \begin{description}
    \item[Motivation for conducting the replication] as a 
description of why the replication was conducted: 
to validate the results, to broaden the results by 
changing the participant pool or the artifacts. 
    \item[Level of interaction with original experimenters.]
The level of interaction between the original experimenters and the
replicators should be reported. This interaction could range from none (i.e.
simply read the  paper) to them being the same people. There is quite a lot of
discussion of the level of interaction allowed for the replication to be
``successful'', but this level should be reported even without  addressing
the controversy.
    \item[Changes to the original experiment.] Any changes made to the
design, participants, artifacts, procedures, data collected and/or analysis
techniques should be  discussed along with the motivation for the change.
    \end{description}
  \item[Comparison of results to original]~\\
    \begin{description}
    \item[Consistent results,] when replication results supported 
results from the original study, and 
    \item[Differences in results,] when results from the replication 
did not coincide with the results from the original study. 
Authors should also discuss how changes made to the 
experimental design (see above) may have caused 
these differences. 
    \end{description}
    \item[Drawing conclusions across studies]
\end{description}

NB: this section contains portions of text repeated directly from Carver~\cite{Carver10} and 
only slightly massaged. Do not do this for your thesis, write your own thoughts down.

\section{\LaTeX\ details}

\subsection{Environments}

A \LaTeX\ environment is something with opening and closing tags, which look
like \cmd{begin}\{\texttt{name}\} and \cmd{end}\{\texttt{name}\}. Some useful
environments to know:

\begin{center}
\begin{tabular}{ll}
  \texttt{itemize}      & bullet lists\\
  \texttt{enumerate}    & numbered lists\\
  \texttt{description}  & definition lists\\
  \hline
  \texttt{center}       & centered line elements\\
  \texttt{flushright}   & right aligned lines\\
  \texttt{flushleft}    & left aligned lines\\
  \hline
  \texttt{tabular}      & table\\
  \texttt{longtable}    & multi-page table (needs the \texttt{longtable} package)\\
  \texttt{sideways}     & rotates some text\\
  \texttt{quote}        & block quote\\
  \texttt{verbatim}     & unformatted text\\
  \texttt{minipage}     & compound box with elements inside\\
  \texttt{boxedminipage}& compound box with elements inside and a border around it\\
  \hline
  \texttt{table}        & floating table (needs to have \texttt{tabular} nested inside)\\
  \texttt{figure}       & floating figure\\
  \texttt{sourcecode}   & floating listing\\
  \hline
  \texttt{equation}     & mathematical equation\\
  \texttt{lstlisting}   & pretty-printed syntax highligted listing\\
  \texttt{multline}     & mathematical equation spanning over multiple lines\\
  \texttt{eqnarray}     & system of mathematical equations\\
  \texttt{gather}       & bundled mathematical equations\\
  \texttt{align}        & bundled and aligned mathematical equations\\
  \texttt{array}        & matrix\\
  \texttt{CD}           & commutative diagrams\\
\end{tabular}
\end{center}

\section{Listings}

\begin{sourcecode}
\begin{lstlisting}[language=prolog]
define(Ps1,G1,G2)
 :-
    usedNs(G1,Uses),
    ps2n(Ps1,N),
    require(
      member(N,Uses),
      'Nonterminal ~q must not be fresh.',
      [N]),
    new(Ps1,N,G1,G2),
    !.
\end{lstlisting}
\caption{Code in Prolog}
\end{sourcecode}

\begin{sourcecode}
\begin{lstlisting}[language=sdf]
module Syntax

imports Numbers
imports basic/Whitespace

exports
  sorts
    Program Function Expr Ops Name Newline

  context-free syntax
    Function+                          -> Program
    Name Name+ "=" Expr Newline+       -> Function
    Expr Ops Expr                      -> Expr      {left,prefer,cons(binary)}
    Name Expr+                         -> Expr      {avoid,cons(apply)}
    "if" Expr "then" Expr "else" Expr  -> Expr      {cons(ifThenElse)}
    "(" Expr ")"                       -> Expr      {bracket}
    Name                               -> Expr      {cons(argument)}
    Int                                -> Expr      {cons(literal)}
    "-"                                -> Ops       {cons(minus)}
    "+"                                -> Ops       {cons(plus)}
    "=="                               -> Ops       {cons(equal)}
\end{lstlisting}
\caption{Code in SDF}
\end{sourcecode}

\begin{sourcecode}
\begin{lstlisting}[language=Java]
import types.*;
import org.antlr.runtime.*;

public class TestEvaluator 
    public static void main(String[] args) throws Exception {

        // Parse file to program
        ANTLRFileStream input = new ANTLRFileStream(args[0]);
        FLLexer lexer = new FLLexer(input);
        CommonTokenStream tokens = new CommonTokenStream(lexer);
        FLParser parser = new FLParser(tokens);
        Program program = parser.program();

        // Parse sample expression
        input = new ANTLRFileStream(args[1]);
        lexer = new FLLexer(input);
        tokens = new CommonTokenStream(lexer);
        parser = new FLParser(tokens);
        Expr expr = parser.expr();

        // Evaluate program
        Evaluator eval = new Evaluator(program);
        int expected = Integer.parseInt(args[2]);
\end{lstlisting}
\caption{Code in Java}
\end{sourcecode}

\begin{sourcecode}
\begin{lstlisting}[style=mono,language=Python]
#!/usr/local/bin/python
# wiki: BGF
import os
import sys
import slpsns
import elementtree.ElementTree as ET

# root::nonterminal* production*
class Grammar:
  def __init__(self):
    self.roots = []
    self.prods = []
  def parse(self,fname):
    self.roots = []
    self.prods = []
    self.xml = ET.parse(fname)
    for e in self.xml.findall('root'):
      self.roots.append(e.text)
    for e in self.xml.findall(slpsns.bgf_('production')):
      prod = Production()
      prod.parse(e)
      self.prods.append(prod)
\end{lstlisting}
\caption{Code in Python}
\end{sourcecode}