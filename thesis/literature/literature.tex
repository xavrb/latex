\chapter{Literature}\label{sec:biblio}

\textsc{Bib}TeX\ is a JSON-like format for bibliographic entries. Encode each
source once as a \textsc{Bib}\TeX\ entry, give it a name and refer to it from
any place in your thesis. The bibliography at the end of the thesis will be
compiled automatically from those entries that are referenced at least once,
it will also be automatically sorted and fancyfied (URLs, DOIs, etc).

DOI is a digital object identifier, it is uniquely and immutably assigned to
any paper published in a well-established journal or conference proceedings
and can be used to refer to it. When used in a browser, it resolves to a
publisher's website where paper can be obtained. Including DOIs in citations
is considered good practice and lets the readers of your thesis get to the
text of the paper in one click. Books do not have DOIs, only ISBNs; some
workshop proceedings and most unofficial publications do not have DOIs. If you
want to get a DOI assigned to your work such as a piece of code, upload it to
\href{http://www.figshare.com}{FigShare}.

Keys in key-value pairs within each \textsc{Bib}\TeX\ entry are never quoted,
values usually are, but can also be included within curly brackets or left as
is, which works fine for numbers (e.g., years). If you want to preserve the
value from any adjustments (e.g., no recapitalisation in titles), use curlies
\emph{and} quotes. Separate authors and editors by ``and'', which will
automatically be mapped to commas or left as ``and''s as necessary.

\section{Books}

\cite{GruneJacobs} is just as good as the Dragon Book, but newer and has an
awesome extended bibliography available for free.

\begin{snippet}
\begin{verbatim}
@book{GruneJacobs,
  author    = "D. Grune and C. J. H. Jacobs",
  title     = "{Parsing Techniques: A Practical Guide}",
  series    = "Monographs in Computer Science",
  edition   = 2,
  publisher = "Springer",
  url       = "http://www.cs.vu.nl/~dick/PT2Ed.html",
  year      = 2008,
}
\end{verbatim}
\end{snippet}

\section{Journal papers}

Not all TOSEM papers are hard to read~\cite{GrammarwareAgenda}.

\begin{snippet}
\begin{verbatim}
@article{GrammarwareAgenda,
  author      = "Paul Klint and Ralf L{\"a}mmel and Chris Verhoef",
  title       = "{Toward an Engineering Discipline for Grammarware}",
  journal     = "ACM Transactions on Software Engineering Methodology (TOSEM)",
  volume      = 14,
  number      = 3,
  year        = 2005,
  pages       = "331--380",
}
\end{verbatim}
\end{snippet}

\section{Conference papers}

There is no limit to how many grammars can be used in one paper, but the
current record stands at 569~\cite{Micropatterns2013}.

\begin{snippet}
\begin{verbatim}
@inproceedings{Micropatterns2013,
  author = "Vadim Zaytsev",
  title = "{Micropatterns in Grammars}",
  booktitle = "{Proceedings of the Sixth International Conference on Software Language Engineering
                (SLE 2013)}",
  year = 2013,
  editor = "Martin Erwig and Richard F. Paige and Eric Van Wyk",
  volume = "8225",
  series = "LNCS",
  pages = "117--136",
  address = "Switzerland",
  month = oct,
  publisher = "Springer International Publishing",
  doi = "10.1007/978-3-319-02654-1_7",
}
\end{verbatim}
\end{snippet}

\section{Theses}

The seventh PhD student of Paul Klint was Jan Rekers~\cite{Rekers92}.

\begin{snippet}
\begin{verbatim}
@phdthesis{Rekers92,
 author   = "J. Rekers",
 title    = "{Parser Generation for Interactive Environments}",
 school   = "University of Amsterdam",
 year     = 1992,
 url      = "http://homepages.cwi.nl/~paulk/dissertations/Rekers.pdf",
}
\end{verbatim}
\end{snippet}

There is also \texttt{mastersthesis} type with exactly the same structure for
referring to Master's theses.

\section{Technical reports}

The original seminal work introducing two-level grammars was never published
in any book or conference, but there is a technical report explaining
it~\cite{Wijngaarden65}. SMC, or \emph{Stichting Matematisch Centrum}, was the
old name of CWI fifty years ago.

\begin{snippet}
\begin{verbatim}
@techreport{Wijngaarden65,
        author      = "Adriaan van Wijngaarden",
        title       = "{Orthogonal Design and Description of a Formal Language}",
        month       = oct,
        year        = 1965,
        institution = "SMC",
        type        = "{MR 76}",
        url         = "http://www.fh-jena.de/~kleine/history/languages/VanWijngaarden-MR76.pdf",
}
\end{verbatim}
\end{snippet}

\section{Wikipedia}

You do not refer to Wikipedia from academic writing, it works the other way around.

\section{Anything else}

You can refer to pretty much anything (websites, blog posts, software) through
\texttt{misc} type of entry~\cite{ANTLR}:

\begin{snippet}
\begin{verbatim}
@misc{ANTLR,
 author       = "Terence Parr",
 title        = "{ANTLR---ANother Tool for Language Recognition}",
 howpublished = "Software",
 url          = "http://antlr.org",
 year         = "2008"
}
\end{verbatim}
\end{snippet}
